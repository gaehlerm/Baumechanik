\documentclass[a4paper, 12pt, titlepage=true]{scrreprt}
\usepackage{graphicx,wrapfig}
\usepackage{subcaption}
\usepackage[ngerman]{varioref}
\usepackage[utf8]{inputenc}
\usepackage{harvard}
\usepackage[ngerman]{babel}
\usepackage[babel,german=swiss]{csquotes}
\usepackage[usenames,dvipsnames]{color}
%\usepackage{fancyhdr}
%\usepackage{rotating}
%\usepackage{multicol}
%\usepackage{floatflt}
\usepackage{enumerate}
%\usepackage{longtable}
%\usepackage{tikz}
%\usepackage{amsmath}
\usepackage[left=25mm,right=25mm,top=25mm,bottom=35mm]{geometry}
%\usepackage{enumerate}
%\usepackage{wrapfig}
%\usepackage{caption}
%\usepackage{gensymb}

\usepackage{sidecap}
\usepackage{amssymb,amsmath}
\DeclareGraphicsExtensions{.png, .jpg, .jpeg, .eps}
\graphicspath{{./images/}}
\setlength{\fboxrule}{4pt}
\setlength{\fboxsep}{20pt}

\definecolor{blau}{cmyk}{1,0,0.,0}
\newcounter{loesung}[chapter]
\newcounter{aufgabe}[chapter]
\renewcommand{\labelenumi}{\arabic{enumi}.}
\renewcommand{\labelenumii}{\arabic{enumi}.\arabic{enumii}}

\newcommand{\aufgabe}{\refstepcounter{aufgabe}
\textbf{\vspace{5pt} \\ Aufgabe \thechapter.\theaufgabe: }}

\newcommand{\loesung}{\refstepcounter{loesung}
\textbf{\vspace{5pt} \\ Loesung \thechapter.\theloesung: }}

\begin{document}

\title{Physik für Baustatiker \\ Loesungen \\ HSLU, HS 2017}
\author{von \\ \bf{Marco Gähler}}
%\maketitle

\textbf{Formelsammlung Baumechanik} (noch nicht komplett und das Design ist noch verbesserungswürdig...)
\\
\textbf{Einheitenvorsätze}
\begin{table}[h]
\centering
\begin{tabular}{ c c c }
 Einheitenvorsatz & Abkürzung & Wert \\
 Tera & T & $10^{12}$ \\
 Giga & G & $10^9$\\
 Mega & M & $10^6$\\
 Kilo & k & $10^3$\\
 Hekto & h & $100$\\
 Dezi & d & $0.1$\\
 Zenti & c & $0.01$\\
 Milli  & m & $10^{-3}$\\
 Mirko & $\mu$ & $10^{-6}$\\
 Nano & n & $10^{-9}$\\
\end{tabular}
%\caption{\textsf{Eine Liste mit den wichtigsten Einheitenvorsätzen.}}
\end{table}

\textbf{Kinematik}
\begin{equation}
s(t) = s_0 + v_0\cdot t + \frac{1}{2}a\cdot t^2.
\end{equation}
\begin{equation}
v(t) = a\cdot \Delta t + v_0
\end{equation}
\begin{equation}
a(t) = a = konst
\end{equation}
\begin{equation}
g \approx 10m/s^2
\end{equation}

\textbf{Kräfte}
\begin{equation}
\vec{F} = m\cdot \vec{a}
\end{equation}
\begin{equation}
\vec{F}_{ab} = -\vec{F}_{ba}
\end{equation}

Reibungskraft $F_R = F_s \cdot \mu$ 

Tabelle mit Reibungskoeffizienten:

\begin{center}
\begin{tabular}{ c c c }
Stoff &	$\mu_{HR}$ & $\mu_{GR}$\\
Stahl auf Stahl & 0.2 &	0.1\\
Stahl auf Holz 	& 0.5 & 0.4\\
Stahl auf Stein & 0.8 & 0.7\\
Stein auf Holz 	& 0.9 & 0.7\\
Leder auf Metall & 0.6 & 0.4\\
Holz auf Holz &	0.5 & 0.4\\
Stein auf Stein & 1.0 & 0.9\\
Stahl auf Eis & 0.03 & 0.01\\
Stahl auf Beton & 0.35 & 0.20\\
Gummi auf Beton (trocken) & 1.0 & 0.8\\
Gummi auf Beton (nass) & 0.3 & 0.25\\
\end{tabular}
\end{center}
\vspace{0.5cm}

Zentripetalkraft $F_z = \frac{v^2}{r} = \dot{\omega}^2\cdot r$

Federkraft $F_f = k\cdot x$

\textbf{Impuls}\\
$p = m\cdot v$\\
$F = \frac{\Delta p}{\Delta t}$\\
Raketengleichung ($R = \frac{\Delta m}{\Delta t}$ pro Zeit ausgestossener Treibstoff)\\

$a(t) = \frac{v_{rel} \cdot R}{m_0-R\cdot t}$\\
$v(t) = v_{rel}\cdot ln\left(\frac{m_0}{m(t)}\right)$


\textbf{Energie}\\
$E_{pot} = mgh$\\
$W = \vec{F}\vec{s} = F\cdot cos(\alpha)\cdot s$\\
$E_{kin} = \frac{1}{2}mv^2$




\vspace{0.5cm}
\textbf{Kreisbewegung}
$x = r\cdot cos(\omega)$\\
$y = r\cdot sin(\omega)$\\
$s = r\cdot \omega$ (in Radiant!!)\\
$\omega [rad] = \frac{2\pi}{360} \omega [^\circ]$



\end{document}